\documentclass[11pt,twoside,english]{scrartcl}
\usepackage{amsmath,amsfonts,amssymb}
\usepackage{textcomp,listings,fancyhdr,babel}
\usepackage[utf8]{inputenc}
\usepackage{graphicx}

\newcommand{\sintef}  {\textsc{sintef}}
\newcommand{\SINTEF}  {\textsc{SINTEF}}
\newcommand{\SAM}     {{\SINTEF} Applied Mathematics}
\newcommand{\svnrev}  {${}$Revision: 1423 ${}$}
\newcommand{\svndate} {${}$Date: 2009-02-23 15:01:00 +0100 (ma, 23 feb 2009) ${}$}
\newcommand{\matlab}  {\textsc{matlab}}
\newcommand{\MATLAB}  {\textsc{MATLAB}}
\newcommand{\rstroot} {\texttt{\$RSTROOT}}
\newcommand{\matlabtm}{\matlab\texttrademark}
\newcommand{\Emacs}   {\textsc{Emacs}}

% Various definitions pertaining to MATLAB
%
\newcommand{\MATLABKeyword}[1]{\texttt{#1}}
\newcommand{\FileName}     [1]{\texttt{#1}}
\newcommand{\FunctionName} [1]{\texttt{#1}}
\newcommand{\function}        {\MATLABKeyword{function}}

\pagestyle{fancy}
\lstset{upquote}

\fancyhead{}
\fancyhead[RO,LE]{\thepage}
\fancyhead[LO,RE]{Release Process}

\fancyfoot{}
\fancyfoot[LE]{\svnrev}
\fancyfoot[LO,R]{}

% Don't number any section headings.
\setcounter{secnumdepth}{-2}

\title {Release Process for\\{\MATLAB} Reservoir Simulation Toolbox}
\author{{\SAM}}
\date{\svndate}

\begin{document}
\maketitle\thispagestyle{empty}
This document describes the release process of the {\MATLAB} Reservoir Simulation Toolbox (RST) developed
by {\SINTEF} Applied Mathematics (SAM).  

%We do not seek to codify and
%enforce very detailed and rigorous standards.  However, in the interest
%of achieving a consistent and readable code base, we nevertheless
%highlight a few conventions which all developers should follow.  In this
%document, the notation {\rstroot} refers to the absolute pathname
%(directory name) of the directory in which the toolbox is physically
%installed on a computer.


\section{Goal}
Two-phase incompressible pressure and transport - both fine scale and
a simple version of multiscale. 


\section{Plan}


\begin{enumerate}

\item Implement needed functionality.

\item Make examples

\item Clean up - comment difficult parts of the code

\item Test against other simulators

\item Make tutorial/documentation

\end{enumerate}

\section{Comments from {\MATLAB} meeting 09.01.09}

\subsection{NEED}

\begin{itemize}
\item Write simple detailed documentation with simple examples:
\begin{itemize}
\item how to construct a grid
\item how to initialize a permeability field
\item how to construct the system (and so on...). 
\end{itemize}

\item Make example that shows how the ``complicated'' functions like
\textbf{solveIncompFlow} can be replaced by slower, but simpler code. Write a
VERY simple simulator example that uses the least possible number of
black-box functions, but refer to the black-box function in each step
for use in more complicated cases (i.e example with direct
simulation). The simple examples can be put in a folder called
tutorial.

\item Include a simple version of the multiscale simulator in the
  release. This version should not use wells due to possibility of large
  errors for particular cases of well placements. Remove possibilities
  for user to set up cases that will not work! For instance only allow
  2-D with simple boundary conditions. 

\item Write about limitations in the documentation.

\item The code should be published in a way that allows us to collect
  the e-mail addresses of the downloaders. For instance we can demand
  that the user must give a valid e-mail address where he will receive a
password for downloading. 
\end{itemize}

\subsection{Future Development}

\begin{itemize}
\item Implement a 2-point flux solver (directly, not through
mimetic inner prod). This however is not straight forward due to
treatment of wells and boundary conditions. 

\item Publish a book about the simulator?

\item UiB: incorporate their code into our simulator. They have a nice
multiscale implementation (more complex than ours, with
compressibility), but only for simple grids. Possibly also MPFA?

\item Add more physics (our contribution is on the numerical part, especially
the grid) - flash toolbox? Corporate with UiB?

\item Compatibility with Octave. 


\end{itemize}


\section{Comments from {\MATLAB} meeting 20.02.09}

Release date: mid march. Release as beta-version first, publish paper
later, perhaps book.
 

\begin{itemize}

\item Testing: Knut-Andreas was elected test pilot. 
GPRS is not available yet for comparison. For now, make tests with analytical solutions,
linear pressure drop and so on. 


\item 
Remove use of single-precision, or make a computeGeometryS that uses
single. 


\item
Remove default behaviour - default saturations, both s and z in
resSol. Code should not do anything else than what the user asks it to
do. 

\item
GPL-licence is probably the best alternative. Talk to Trond about
their server. Could probably use a standard program for sending password to email. 

\item
Add teasers, examples of black-oil, multiscale ++. Available upon
request. 

\end{itemize}

\subsection{System demands and compatibility}

\begin{itemize}

\item
New (unreleased) version of Octave supports bsxfun ++. Should be a
small job to make the simulator compatible with this version. Could be
possible to get funding from Statoil to do this. 

\item
Need at least version 7.4 of Matlab to run the simulator. Bsxfun, advanced nested
functions, function pointers and so on. But some features can be used
in earlier versions, for example read and build grid. In addition, it
is possible to download some functions (like bsxfun) from Mathworks
for older versions. 
\end{itemize}







\end{document}

